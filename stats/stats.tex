% Options for packages loaded elsewhere
\PassOptionsToPackage{unicode}{hyperref}
\PassOptionsToPackage{hyphens}{url}
%
\documentclass[
]{article}
\usepackage{amsmath,amssymb}
\usepackage{lmodern}
\usepackage{ifxetex,ifluatex}
\ifnum 0\ifxetex 1\fi\ifluatex 1\fi=0 % if pdftex
  \usepackage[T1]{fontenc}
  \usepackage[utf8]{inputenc}
  \usepackage{textcomp} % provide euro and other symbols
\else % if luatex or xetex
  \usepackage{unicode-math}
  \defaultfontfeatures{Scale=MatchLowercase}
  \defaultfontfeatures[\rmfamily]{Ligatures=TeX,Scale=1}
\fi
% Use upquote if available, for straight quotes in verbatim environments
\IfFileExists{upquote.sty}{\usepackage{upquote}}{}
\IfFileExists{microtype.sty}{% use microtype if available
  \usepackage[]{microtype}
  \UseMicrotypeSet[protrusion]{basicmath} % disable protrusion for tt fonts
}{}
\makeatletter
\@ifundefined{KOMAClassName}{% if non-KOMA class
  \IfFileExists{parskip.sty}{%
    \usepackage{parskip}
  }{% else
    \setlength{\parindent}{0pt}
    \setlength{\parskip}{6pt plus 2pt minus 1pt}}
}{% if KOMA class
  \KOMAoptions{parskip=half}}
\makeatother
\usepackage{xcolor}
\IfFileExists{xurl.sty}{\usepackage{xurl}}{} % add URL line breaks if available
\IfFileExists{bookmark.sty}{\usepackage{bookmark}}{\usepackage{hyperref}}
\hypersetup{
  pdftitle={StatisticsForRoboticSystemsA2},
  pdfauthor={Sergio},
  hidelinks,
  pdfcreator={LaTeX via pandoc}}
\urlstyle{same} % disable monospaced font for URLs
\usepackage[margin=1in]{geometry}
\usepackage{color}
\usepackage{fancyvrb}
\newcommand{\VerbBar}{|}
\newcommand{\VERB}{\Verb[commandchars=\\\{\}]}
\DefineVerbatimEnvironment{Highlighting}{Verbatim}{commandchars=\\\{\}}
% Add ',fontsize=\small' for more characters per line
\usepackage{framed}
\definecolor{shadecolor}{RGB}{248,248,248}
\newenvironment{Shaded}{\begin{snugshade}}{\end{snugshade}}
\newcommand{\AlertTok}[1]{\textcolor[rgb]{0.94,0.16,0.16}{#1}}
\newcommand{\AnnotationTok}[1]{\textcolor[rgb]{0.56,0.35,0.01}{\textbf{\textit{#1}}}}
\newcommand{\AttributeTok}[1]{\textcolor[rgb]{0.77,0.63,0.00}{#1}}
\newcommand{\BaseNTok}[1]{\textcolor[rgb]{0.00,0.00,0.81}{#1}}
\newcommand{\BuiltInTok}[1]{#1}
\newcommand{\CharTok}[1]{\textcolor[rgb]{0.31,0.60,0.02}{#1}}
\newcommand{\CommentTok}[1]{\textcolor[rgb]{0.56,0.35,0.01}{\textit{#1}}}
\newcommand{\CommentVarTok}[1]{\textcolor[rgb]{0.56,0.35,0.01}{\textbf{\textit{#1}}}}
\newcommand{\ConstantTok}[1]{\textcolor[rgb]{0.00,0.00,0.00}{#1}}
\newcommand{\ControlFlowTok}[1]{\textcolor[rgb]{0.13,0.29,0.53}{\textbf{#1}}}
\newcommand{\DataTypeTok}[1]{\textcolor[rgb]{0.13,0.29,0.53}{#1}}
\newcommand{\DecValTok}[1]{\textcolor[rgb]{0.00,0.00,0.81}{#1}}
\newcommand{\DocumentationTok}[1]{\textcolor[rgb]{0.56,0.35,0.01}{\textbf{\textit{#1}}}}
\newcommand{\ErrorTok}[1]{\textcolor[rgb]{0.64,0.00,0.00}{\textbf{#1}}}
\newcommand{\ExtensionTok}[1]{#1}
\newcommand{\FloatTok}[1]{\textcolor[rgb]{0.00,0.00,0.81}{#1}}
\newcommand{\FunctionTok}[1]{\textcolor[rgb]{0.00,0.00,0.00}{#1}}
\newcommand{\ImportTok}[1]{#1}
\newcommand{\InformationTok}[1]{\textcolor[rgb]{0.56,0.35,0.01}{\textbf{\textit{#1}}}}
\newcommand{\KeywordTok}[1]{\textcolor[rgb]{0.13,0.29,0.53}{\textbf{#1}}}
\newcommand{\NormalTok}[1]{#1}
\newcommand{\OperatorTok}[1]{\textcolor[rgb]{0.81,0.36,0.00}{\textbf{#1}}}
\newcommand{\OtherTok}[1]{\textcolor[rgb]{0.56,0.35,0.01}{#1}}
\newcommand{\PreprocessorTok}[1]{\textcolor[rgb]{0.56,0.35,0.01}{\textit{#1}}}
\newcommand{\RegionMarkerTok}[1]{#1}
\newcommand{\SpecialCharTok}[1]{\textcolor[rgb]{0.00,0.00,0.00}{#1}}
\newcommand{\SpecialStringTok}[1]{\textcolor[rgb]{0.31,0.60,0.02}{#1}}
\newcommand{\StringTok}[1]{\textcolor[rgb]{0.31,0.60,0.02}{#1}}
\newcommand{\VariableTok}[1]{\textcolor[rgb]{0.00,0.00,0.00}{#1}}
\newcommand{\VerbatimStringTok}[1]{\textcolor[rgb]{0.31,0.60,0.02}{#1}}
\newcommand{\WarningTok}[1]{\textcolor[rgb]{0.56,0.35,0.01}{\textbf{\textit{#1}}}}
\usepackage{graphicx}
\makeatletter
\def\maxwidth{\ifdim\Gin@nat@width>\linewidth\linewidth\else\Gin@nat@width\fi}
\def\maxheight{\ifdim\Gin@nat@height>\textheight\textheight\else\Gin@nat@height\fi}
\makeatother
% Scale images if necessary, so that they will not overflow the page
% margins by default, and it is still possible to overwrite the defaults
% using explicit options in \includegraphics[width, height, ...]{}
\setkeys{Gin}{width=\maxwidth,height=\maxheight,keepaspectratio}
% Set default figure placement to htbp
\makeatletter
\def\fps@figure{htbp}
\makeatother
\setlength{\emergencystretch}{3em} % prevent overfull lines
\providecommand{\tightlist}{%
  \setlength{\itemsep}{0pt}\setlength{\parskip}{0pt}}
\setcounter{secnumdepth}{-\maxdimen} % remove section numbering
\ifluatex
  \usepackage{selnolig}  % disable illegal ligatures
\fi

\title{StatisticsForRoboticSystemsA2}
\author{Sergio}
\date{12/11/2022}

\begin{document}
\maketitle

\hypertarget{libraries}{%
\subsection{Libraries}\label{libraries}}

\begin{Shaded}
\begin{Highlighting}[]
\FunctionTok{library}\NormalTok{(readxl)}
\CommentTok{\#library(magrittr)}
\CommentTok{\#library(tidyverse)}
\CommentTok{\#library(rstatix)}
\CommentTok{\#library(ggpubr )}
\FunctionTok{library}\NormalTok{(PMCMRplus)}
\end{Highlighting}
\end{Shaded}

\begin{verbatim}
## Warning: package 'PMCMRplus' was built under R version 4.1.2
\end{verbatim}

\hypertarget{import-data}{%
\subsection{Import data}\label{import-data}}

Firs we import our data. Three outcomes variables were measured: time,
distance, and kinematic estimation error (kte). Four conditions were
tested: Adaptive Waypoint with Non-linear control (AN), Adaptive
Waypoint with Proportional control (AP), Fixed Waypoint with Non-linear
control (FN), Fixed Waypoint with Proportional control (FP).

\begin{Shaded}
\begin{Highlighting}[]
\NormalTok{data }\OtherTok{\textless{}{-}} \FunctionTok{read\_excel}\NormalTok{(}\StringTok{"data.xlsx"}\NormalTok{)}
\NormalTok{data}\SpecialCharTok{$}\NormalTok{id }\OtherTok{\textless{}{-}} \FunctionTok{as.factor}\NormalTok{(data}\SpecialCharTok{$}\NormalTok{id)}
\NormalTok{data}\SpecialCharTok{$}\NormalTok{condition }\OtherTok{\textless{}{-}} \FunctionTok{as.factor}\NormalTok{(data}\SpecialCharTok{$}\NormalTok{condition)}
\FunctionTok{summary}\NormalTok{(data)}
\end{Highlighting}
\end{Shaded}

\begin{verbatim}
##        id     condition      time          distance         kte       
##  1      : 4   AN:10     Min.   :21.00   Min.   :1823   Min.   :20.89  
##  2      : 4   AP:10     1st Qu.:23.00   1st Qu.:2263   1st Qu.:25.28  
##  3      : 4   FN:10     Median :24.00   Median :2286   Median :34.23  
##  4      : 4   FP:10     Mean   :24.80   Mean   :2284   Mean   :36.45  
##  5      : 4             3rd Qu.:25.25   3rd Qu.:2342   3rd Qu.:47.37  
##  6      : 4             Max.   :31.00   Max.   :2421   Max.   :58.04  
##  (Other):16
\end{verbatim}

\hypertarget{normality-tests}{%
\subsection{Normality tests}\label{normality-tests}}

We carry normality tests. Shapiro Wilk's test was used due to the small
sample size. The null hypothesis for this test is that the sample
distribution is normal.

\begin{Shaded}
\begin{Highlighting}[]
\NormalTok{norm\_time }\OtherTok{\textless{}{-}} \FunctionTok{shapiro.test}\NormalTok{(data}\SpecialCharTok{$}\NormalTok{time)}
\NormalTok{distance }\OtherTok{\textless{}{-}} \FunctionTok{shapiro.test}\NormalTok{(data}\SpecialCharTok{$}\NormalTok{distance)}
\NormalTok{kte }\OtherTok{\textless{}{-}} \FunctionTok{shapiro.test}\NormalTok{(data}\SpecialCharTok{$}\NormalTok{kte)}

\FunctionTok{print}\NormalTok{(norm\_time)}
\end{Highlighting}
\end{Shaded}

\begin{verbatim}
## 
##  Shapiro-Wilk normality test
## 
## data:  data$time
## W = 0.81886, p-value = 1.717e-05
\end{verbatim}

\begin{Shaded}
\begin{Highlighting}[]
\FunctionTok{print}\NormalTok{(distance)}
\end{Highlighting}
\end{Shaded}

\begin{verbatim}
## 
##  Shapiro-Wilk normality test
## 
## data:  data$distance
## W = 0.75462, p-value = 8.703e-07
\end{verbatim}

\begin{Shaded}
\begin{Highlighting}[]
\FunctionTok{print}\NormalTok{(kte)}
\end{Highlighting}
\end{Shaded}

\begin{verbatim}
## 
##  Shapiro-Wilk normality test
## 
## data:  data$kte
## W = 0.87698, p-value = 0.0004367
\end{verbatim}

\hypertarget{friedman-tests}{%
\subsection{Friedman tests}\label{friedman-tests}}

Last tests rejected null hypothesis, hence Friedman is applied instead
of ANOVA. This tests was chosen to assess differences between multiple
conditions, with matched samples (i.e., same robot), and
continuous/interval outcome variable. The null hyphotesis is that all
conditions are equal.

\begin{Shaded}
\begin{Highlighting}[]
\NormalTok{friedman\_time }\OtherTok{\textless{}{-}} \FunctionTok{friedman.test}\NormalTok{(data}\SpecialCharTok{$}\NormalTok{time, data}\SpecialCharTok{$}\NormalTok{condition, data}\SpecialCharTok{$}\NormalTok{id )}
\NormalTok{friedman\_distance }\OtherTok{\textless{}{-}} \FunctionTok{friedman.test}\NormalTok{(data}\SpecialCharTok{$}\NormalTok{distance, data}\SpecialCharTok{$}\NormalTok{condition, data}\SpecialCharTok{$}\NormalTok{id )}
\NormalTok{friedman\_kte }\OtherTok{\textless{}{-}} \FunctionTok{friedman.test}\NormalTok{(data}\SpecialCharTok{$}\NormalTok{kte, data}\SpecialCharTok{$}\NormalTok{condition, data}\SpecialCharTok{$}\NormalTok{id )}

\FunctionTok{print}\NormalTok{(friedman\_time)}
\end{Highlighting}
\end{Shaded}

\begin{verbatim}
## 
##  Friedman rank sum test
## 
## data:  data$time, data$condition and data$id
## Friedman chi-squared = 22.291, df = 3, p-value = 5.675e-05
\end{verbatim}

\begin{Shaded}
\begin{Highlighting}[]
\FunctionTok{print}\NormalTok{(friedman\_distance)}
\end{Highlighting}
\end{Shaded}

\begin{verbatim}
## 
##  Friedman rank sum test
## 
## data:  data$distance, data$condition and data$id
## Friedman chi-squared = 15, df = 3, p-value = 0.001817
\end{verbatim}

\begin{Shaded}
\begin{Highlighting}[]
\FunctionTok{print}\NormalTok{(friedman\_kte)}
\end{Highlighting}
\end{Shaded}

\begin{verbatim}
## 
##  Friedman rank sum test
## 
## data:  data$kte, data$condition and data$id
## Friedman chi-squared = 25.56, df = 3, p-value = 1.179e-05
\end{verbatim}

\hypertarget{posthoc-tests}{%
\subsection{Posthoc tests}\label{posthoc-tests}}

Last tests rejected null hypothesis. Posthoc test are required for
pairwise comparisons between conditions. The Bonferroni correction was
chosen, as it is the standard for this test in literature.

\begin{Shaded}
\begin{Highlighting}[]
\NormalTok{posthoc\_time }\OtherTok{=} \FunctionTok{frdAllPairsConoverTest}\NormalTok{(}\AttributeTok{y=}\NormalTok{data}\SpecialCharTok{$}\NormalTok{time, }\AttributeTok{groups=}\NormalTok{data}\SpecialCharTok{$}\NormalTok{condition, }\AttributeTok{blocks=}\NormalTok{data}\SpecialCharTok{$}\NormalTok{id, }\AttributeTok{p.adjust=}\StringTok{"bonferroni"}\NormalTok{)}
\NormalTok{posthoc\_distance }\OtherTok{=} \FunctionTok{frdAllPairsConoverTest}\NormalTok{(}\AttributeTok{y=}\NormalTok{data}\SpecialCharTok{$}\NormalTok{distance, }\AttributeTok{groups=}\NormalTok{data}\SpecialCharTok{$}\NormalTok{condition, }\AttributeTok{blocks=}\NormalTok{data}\SpecialCharTok{$}\NormalTok{id, }\AttributeTok{p.adjust=}\StringTok{"bonferroni"}\NormalTok{)}
\NormalTok{posthoc\_kte }\OtherTok{=} \FunctionTok{frdAllPairsConoverTest}\NormalTok{(}\AttributeTok{y=}\NormalTok{data}\SpecialCharTok{$}\NormalTok{kte, }\AttributeTok{groups=}\NormalTok{data}\SpecialCharTok{$}\NormalTok{condition, }\AttributeTok{blocks=}\NormalTok{data}\SpecialCharTok{$}\NormalTok{id, }\AttributeTok{p.adjust=}\StringTok{"bonferroni"}\NormalTok{)}

\FunctionTok{print}\NormalTok{(posthoc\_time)}
\end{Highlighting}
\end{Shaded}

\begin{verbatim}
## 
##  Pairwise comparisons using Conover's all-pairs test for a two-way balanced complete block design
\end{verbatim}

\begin{verbatim}
## data: y, groups and blocks
\end{verbatim}

\begin{verbatim}
##    AN     AP     FN    
## AP 0.0297 -      -     
## FN 1.0000 0.0903 -     
## FP 0.0035 1.0000 0.0117
\end{verbatim}

\begin{verbatim}
## 
## P value adjustment method: bonferroni
\end{verbatim}

\begin{Shaded}
\begin{Highlighting}[]
\FunctionTok{print}\NormalTok{(posthoc\_distance)}
\end{Highlighting}
\end{Shaded}

\begin{verbatim}
## 
##  Pairwise comparisons using Conover's all-pairs test for a two-way balanced complete block design
## 
## data: y, groups and blocks
\end{verbatim}

\begin{verbatim}
##    AN     AP     FN    
## AP 0.4176 -      -     
## FN 0.0774 1.0000 -     
## FP 0.0059 0.4947 1.0000
\end{verbatim}

\begin{verbatim}
## 
## P value adjustment method: bonferroni
\end{verbatim}

\begin{Shaded}
\begin{Highlighting}[]
\FunctionTok{print}\NormalTok{(posthoc\_kte)}
\end{Highlighting}
\end{Shaded}

\begin{verbatim}
## 
##  Pairwise comparisons using Conover's all-pairs test for a two-way balanced complete block design
## 
## data: y, groups and blocks
\end{verbatim}

\begin{verbatim}
##    AN     AP     FN    
## AP 1.0000 -      -     
## FN 0.0977 0.0078 -     
## FP 0.0187 0.0013 1.0000
\end{verbatim}

\begin{verbatim}
## 
## P value adjustment method: bonferroni
\end{verbatim}

\end{document}
